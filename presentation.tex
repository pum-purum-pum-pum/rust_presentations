% Latex template: mahmoud.s.fahmy@students.kasralainy.edu.eg
% For more details: https://www.sharelatex.com/learn/Beamer

\documentclass{beamer}
\usetheme{default}

\usepackage{minted}
\usepackage{hyperref}
% \hypersetup{colorlinks=true}



% THEME DARK
% \usetheme{Default}

\setbeamercolor{normal text}{fg=white,bg=black!90}
\setbeamercolor{structure}{fg=white}

\setbeamercolor{alerted text}{fg=red!85!black}

\setbeamercolor{item projected}{use=item,fg=black,bg=item.fg!35}

\setbeamercolor*{palette primary}{use=structure,fg=structure.fg}
\setbeamercolor*{palette secondary}{use=structure,fg=structure.fg!95!black}
\setbeamercolor*{palette tertiary}{use=structure,fg=structure.fg!90!black}
\setbeamercolor*{palette quaternary}{use=structure,fg=structure.fg!95!black,bg=black!80}

\setbeamercolor*{framesubtitle}{fg=white}

\setbeamercolor*{block title}{parent=structure,bg=black!60}
\setbeamercolor*{block body}{fg=black,bg=black!10}
\setbeamercolor*{block title alerted}{parent=alerted text,bg=black!15}
\setbeamercolor*{block title example}{parent=example text,bg=black!15}



\AtBeginSection[]{
  \begin{frame}
  \vfill
  \centering
  \begin{beamercolorbox}[sep=8pt,center,shadow=true,rounded=true]{title}
    \usebeamerfont{title}\insertsectionhead\par%
  \end{beamercolorbox}
  \vfill
  \end{frame}
}
% THEME DARK


\usemintedstyle{paraiso-dark}
\newminted{rust}{fontsize=\scriptsize, 
		  %  linenos,
		   numbersep=8pt,
		   gobble=4,
		   frame=lines,
		   bgcolor=bg,
framesep=3mm} 

    \title{My experience of using Rust for game development}	% Presentation title
    \author{Vlad Zhukov}
    % \author[shortname]{author1  \and author2 }
    % \institute[shortinst]{\inst{1} asd for author1 \and %
    %                   \inst{2} affiliation for author2}
    % Presentation author								% Today's date	
    \date{}
    \begin{document}
    \definecolor{bg}{rgb}{0.1,0.1,0.1}
    % Title page
    % This page includes the informations defined earlier including title, author/s, affiliation/s and the date
    \begin{frame}
      \titlepage
    \end{frame}
    
    % Outline
    % This page includes the outline (Table of content) of the presentation. All sections and subsections will appear in the outline by default.
    \begin{frame}{Content}
      \tableofcontents
    \end{frame}

    \section{Game}
    \begin{frame}
      \title{Game}
      \begin{itemize}
        \item 3D world with 2D graphics and 2D gamplay
        \item Slices of 3D world with 2D mechanics
      \end{itemize}
    \end{frame}

    % \begin{frame}
    %   \begin{figure}
    %     \centering
    %     \includegraphics[width=1.0\linewidth]{editor_screenshot.png}
    %     \caption{Editor screenshot}
    %     \label{fig:seq2seq}
    %   \end{figure}
    %   \begin{itemize}
    %     \item Free images are used 
    %     \item Game will look completely different(much better :))
    %   \end{itemize}
    % \end{frame}

    % \begin{frame}
    %   \begin{figure}
    %     \centering
    %     \includegraphics[width=1.0\linewidth]{gameplay_screenshot.png}
    %     \caption{gameplay screenshot}
    %     \label{fig:seq2seq}
    %   \end{figure}
    %   \begin{itemize}
    %     \item Free images are used 
    %     \item Game will look completely different(much better :))
    %   \end{itemize}
    % \end{frame}

    \defverbatim[colored]\structure{
      \begin{rustcode}
        crate
        └── engine
            ├── events
            ├── game_logic
            ├── graphics
            │   ├── ...
            ├── map
            │   ├── ...
            ├── npc
            │   ├── ...
            ├── physics
            └── ui 
                ├── ...
      \end{rustcode}
    }

    \section{Project structure}
    \begin{frame}{Main crate structure}
        % \begin{figure}
        %     \centering
        %     \includegraphics[width=1.0\linewidth]{project_structure.pdf}
        %     \caption{Result of \textbf{cargo modules graph}}
        %     \label{fig:seq2seq}
        %   \end{figure}
        \structure
    \end{frame}

    \section{Graphics}
    \defverbatim[colored]\Renderer{
    Glium with specs
    \begin{rustcode}
      // gfx_system.rs
      #[repr(transparent)]
      pub struct Renderer<'a> {
          pub data: RenderingData<'a>, 
                 // &RenderingData is often used as a parameter
      }
      
      impl<'a, 'b> System<'a> for Renderer<'b> {
          ...
        fn run(&mut self, data: Self::SystemData) {
          let mut target = self.data.display.draw(); 
                                  // display: &'a glium::Display,
          <DRAW CALLS HERE>
          <CONROD GUI EVENTS HANDLING(AND DRAWING)>
          ...
          target.finish().unwrap();
        }
        ..
      }
      \end{rustcode}
    }
    \defverbatim[colored]\RenderingData{
      \begin{rustcode}
        pub struct RenderingData<'a> {
          pub positions_buffers: Vec<glium::VertexBuffer<Vertex3>>,
          pub normals_buffers: Vec<glium::VertexBuffer<Normal3>>,
          pub indices_buffers: Vec<glium::IndexBuffer<u16>>,
          pub textures: Vec<glium::texture::SrgbTexture2d>,
          ...
        }
      }

      \end{rustcode}
    }

    \defverbatim[colored]\RendererDispatch{
      \begin{rustcode}
        // main.rs
        let mut rendering_system = Renderer::new(..)
        let mut dispatcher = DispatcherBuilder::new()
          .with(..)
          ...
          .with_thread_local(rendering_system)
      \end{rustcode}
    }


    \defverbatim[colored]\nphysics{
      Solution by thiolliere 
      \href{https://github.com/thiolliere/airjump-multi}{airjump-multi(GitHub)}
      \begin{rustcode}
        // shared_physics.rs
        pub fn safe_insert<'a>(
            entity: ::specs::Entity,
            // position, inertia, ...
            bodies_handle: &mut ::specs::WriteStorage<'a, ::component::RigidBody>,
            physic_world: &mut ::resource::PhysicWorld,
            bodies_map: &mut ::resource::BodiesMap,
        ) -> Self {
            let body_handle = physic_world.add_rigid_body(position, inertia ...);
            bodies_map.insert(body_handle, entity);
            bodies_handle.insert(entity, RigidBody(body_handle));
            RigidBody(body_handle)
        }
      \end{rustcode}
    }
    \defverbatim[colored]\safemaintain{
      Solution by thiolliere 
      \href{https://github.com/thiolliere/airjump-multi}{airjump-multi(GitHub)}
      \begin{rustcode}
          // main.rs
          pub fn safe_maintain(world: &mut specs::World) {
            world.maintain();
            let mut physic_world = world.write_resource::<World<f32>>();
            let mut bodies_map = world.write_resource::<BodiesMap>();
        
            let retained = world
                .write_storage::<PhysicComponent>()
                .retained()
                .iter()
                .map(|r| r.body_handle)
                .collect::<Vec<_>>();
            physic_world.remove_bodies(&retained);
            for handle in &retained {
                bodies_map.remove(handle);
            }
        }      
      \end{rustcode}
    }
    
    \defverbatim[colored]\IngameUI{
      \begin{rustcode}
        // shared_game_logic.rs
        #[derive(Default, Debug)]
        pub struct IngameUI {
            pub current_draw_primitives: Vec<DrawPrimitive>,
            pub mouse_position: (f32, f32),
            // * shift between mouse end object position when starting moving
            pub moving_shift: Option<Vector3<f32>>,
            // * position of mouse in space
            pub moving_position: Option<Vector3<f32>>, 
                              // Option easier for impl default
            // * state of UI entities interation
            pub hover_id: Option<usize>,
            pub selected_id: Option<usize>,
            pub moving_id: Option<usize>,

            // * state of entities interction
            pub entity_hover_id: Option<::specs::Entity>,
            pub entity_selected_id: Option<::specs::Entity>,
            pub entity_moving_id: Option<::specs::Entity>,
        }
      \end{rustcode}
    }

    \defverbatim[colored]\DrawPrimitives{
      \begin{rustcode}
        #[derive(Debug)]
        pub enum DrawPrimitive {
            Cube{
                position: Vector3<f32>,
                size: f32,
            },
        }

        pub struct Draggable {
            pub position: Vector3<f32>,
        }
      \end{rustcode}
    }

    \defverbatim[colored]\IngameUIExample{
      \begin{rustcode}
        // ingame_ui.rs
        let draggable = Draggable::new(
            Vector3::new(
                current_target_position.x, 
                current_target_position.y, 
                current_target_position.z)
        );
        match draggable.set(
            draggable_id as usize, 
            ingame_ui, // &mut IngameUI
            & system_resources,
            &camera,
            mouse_position
        ) {
            Some(new_position) => {
                // DO THOMETHING WITH NEW VALUE
            }
            None => ()
        }
      \end{rustcode}
    }
    \defverbatim[colored]\RetainedStorage{
      Solution by thiolliere 
      \href{https://github.com/thiolliere/airjump-multi}{airjump-multi(GitHub)}
      \begin{rustcode}
        //retained_storage.rs
        pub struct RetainedStorage<C, T = UnprotectedStorage<C>> {
            retained: Vec<C>,
            storage: T,
            phantom: PhantomData<C>,
        }

        impl<C, T> Retained<C> for RetainedStorage<C, T> {
            fn retained(&mut self) -> Vec<C> {
                mem::replace(&mut self.retained, vec![])
            }
        }
        //shared_physics.rs
        pub struct RigidBody(BodyHandle);
        impl ::specs::Component for RigidBody {
            type Storage = RetainedStorage<Self>;
        }
      \end{rustcode}
    }

    \defverbatim[colored]\Saving{
      map.ron $\leftrightarrow$ Map with RawEntityes $\leftrightarrow$ specs objects
      \begin{rustcode}
        // map/common.rs
        #[derive(Debug, Serialize, Deserialize)]
        pub struct Map {
            pub entities: Vec<RawEntity>,
            pub positions: Vec<Option<Isometry3<f32>>>,
            pub scales: Vec<Option<ScaleComponent>>,
            pub slices: Slices,
        }
      \end{rustcode}
    }
    
    \defverbatim[colored]\RawEntity{
      % \begin{itemize}
      %   \item Builder pattern
      % \end{itemize}
      \begin{rustcode}
        // map/common.rs
        #[derive(Debug, Default, Clone, Serialize, Deserialize, Builder)]
        #[builder(setter(into))]
        #[builder(default)]
        #[builder(derive(Debug, Serialize, Deserialize))]
        pub struct RawEntity {
            pub gl_name: Option<String>,
            pub cs_texture_name: Option<String>,
            pub physic_name: Option<String>,
            pub renderer_name: Option<String>,
            ...
        }
      \end{rustcode}
      Builder pattern with \textit{derive\_builder} crate
    }

    \defverbatim[colored]\OPTLevels{
      \begin{rustcode}
        [profile.dev]
        opt-level = 0
        debug = true

        [profile.dev.overrides."*"]
        opt-level = 3
      \end{rustcode}
    }
    \begin{frame}
      \Renderer
    \end{frame}

    \begin{frame}
      \RenderingData
      it's sort of \ "my little ECS" while using specs...
    \end{frame}

    \begin{frame}
      If any of these please let me know
      \begin{itemize}
        \item mb there is way to implement thread local components/resources with specs
        \item mb there is better way to do it
      \end{itemize}
    \end{frame}

    \begin{frame}
      \RendererDispatch
      glium can be called only thread locally
    \end{frame}

    \section{Physics}

    \begin{frame}
      \begin{itemize}
        \item nphysics
        \item Many systems
        \item One of these calls step()
      \end{itemize}
      How do I use nphysics with specs?
    \end{frame}

    \begin{frame}
      \RetainedStorage
    \end{frame}

    \begin{frame}
      \nphysics
    \end{frame}

    \begin{frame}
      \safemaintain
    \end{frame}

    \section{UI}

    \begin{frame}
      \begin{itemize}
        \item conrod
        \item small "ingame" UI(dragging objects for now)
      \end{itemize}
      conrod is easy, let's talk about ingame UI
    \end{frame}

    \begin{frame}
      ImGUI style (only for editor, easiest implementation)
      \IngameUI
    \end{frame}

    \begin{frame}
      \DrawPrimitives
    \end{frame}

    \begin{frame}
      \IngameUIExample
    \end{frame}

    \section{Map and saving}
    \begin{frame}
      \centering
      How to implement saving?
      \begin{itemize}
        \item specs serde searializing deserializing (Hard)
        \item searializing/deserializing intermidiate objects (Ok)
      \end{itemize}
    \end{frame}

    \begin{frame}
      \Saving
    \end{frame}

    \begin{frame}
      \RawEntity
    \end{frame}

    \section{Compilation time}
    \begin{frame}
      Rust nightly opt-level. My crate is in debug, dependencies are in release
      \OPTLevels
    \end{frame}

    \begin{frame}
      Separate project on crates helps(not that much)\\
      My crates are:
      \begin{itemize}
        \item main crate
        \item misc (probably will name it "common")
        \item physics
        \item rendering
      \end{itemize}
      $\sim$ 10s when change parent crate vs $\sim$ 20s without crate separating
      Don't do it like that
    \end{frame}
\end{document}